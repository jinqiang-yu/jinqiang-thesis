\chapter{Introduction}\label{chap:intro}

\section{Achievements of Machine Learning}

The rapid progress in machine learning~(ML) and Artificial Intelligence~(AI) techniques
has drastically transformed various aspects of human lives over recent years~\cite{bengio-nature15,taward18}.
%
These impressive ML and AI methods, due to their advancements in algorithms
and enhanced computational capabilities,
%and the accessibility of extensive datasets,  
have found vast use across
diverse tasks, such as those in natural language processing~(NLP) and computer vision~(CV) fields.
%domains.

%transportation~\cite{adlb-sust19,nmnk-sust20,hy-rits20}, 
%law~\cite{raai-sp19,ksarv-mecon22}, finance~\cite{aifin20,gklp-jbef21,af-fintech19}, 
%and healthcare~\cite{aimed-jbhi20,mpr-jfmpc19,rcbt-nm22,racc-jamia20,xaihealth-access22}


In the language technology field, a notable advancement has been made through the development
of large language models~(LLMs)~\cite{llm-tist23,gptedu-lid23,transformer-17}, BERT~\cite{dclt-bert19}, GPT-3~\cite{gpt-3-neurips20}, GPT-4~\cite{gpt-4-corr23}, and LLaMA-2~\cite{llama-2-corr23}.
%
These models have showcased remarkable performance across various natural language processing (NLP) tasks.
%
Furthermore, they signify a notable advancement in natural language generation~(NLG) 
and natural language understanding~(NLU) capabilities, 
highlighting their adaptability across diverse applications such as chatbots and content generation.
%
Leading technology companies have integrated LLMs into their commercial products 
and services to augment functionality.
%
As an example, Microsoft integrates GPT into the new Bing to enhance search relevance ranking~\cite{bing-17}.
%
Meanwhile, advancements in the image generation realm, facilitated by technologies 
such as Stable Difussion~\cite{rbleo-cvpr22}, variational autoencoders~(VAE)~\cite{kw-corr13}, generative adversarial networks (GANs)~\cite{gan-corr14},
vision transformers~\cite{vtransformer-21}, have significantly influenced the field,
notably improving image generation capabilities across domains such as computer graphics and artistic design.
%  [87, 302, 307, 378]
%
Moreover, significant milestones have been achieved in the filed of reinforcement learning, 
illustrated by the outstanding performance of AI agents, such as AlphaGo~\cite{rl-nature15,rlchess-science18,rlgo-nature17}.
%
These agents outperformed world champions in complex games like chess and Go,
demonstrating the potential of AI in strategic decision-making.

The significance of ML advancements for society is also highlighted by 
the substantial contributions, particularly in
the general AI/ML field, made by leading technology companies,
such as Amazon, Facebook, Google, Microsoft, Baidu, Amazon, Oracle,
and many others.
%
Furthermore, the emergence of AI for Science~(AI4Science)~\cite{ai4sci-corr23,aiqtcs-corr23}, 
making a bridge to connect AI technologies with scientific fields, 
has proven to be an influential method for accelerating scientific research and discovery.
%
AI4Science leverages AI's computational power to analyze complex scientific data, formulate hypotheses, 
and foster advancements in areas like materials science~\cite{bdciw-nature18}, 
climate science~\cite{khu-natureg20,dbcl-gmd23}, and genomics~\cite{aige-aihl23,wsnp-nrg22}.

While these accomplishments have raised awareness of the transformative potential of ML and AI, 
they have also emphasized the critical importance of dependable and trustworthy
AI~\cite{wing-cacm21,trustai-acmc23,kurd-csur22,sss-cacm22,ms-rw22,msi-fai23}.
%
Maintaining transparency, ethical standards, and responsible implementation are 
crucial to guarantee the positive societal effects of these technologies,
despite their limitless potential.
\pjs{Beware these last three sentences sound like they may be copied from somewhere!}

\section{Rise of eXplainable AI}

Complex ML models are rapidly integrated 
into our daily lives due to the impressive progress in ML.
%
These complex models, driven by their exceptional accuracy, significantly impact decision-making 
across various fields including healthcare~\cite{aimed-jbhi20,mpr-jfmpc19,rcbt-nm22,racc-jamia20,xaihealth-access22},
 law~\cite{raai-sp19,ksarv-mecon22}, finance~\cite{aifin20,gklp-jbef21,af-fintech19}, and
transportation~\cite{adlb-sust19,nmnk-sust20,hy-rits20}.
%
However, accuracy frequently sacrifices explainability, 
leaving both domain experts and general users without clear insights into the outcomes generated 
by these complex model although users need to make decisions based on them.
%
Complex ML models are commonly referred to as ``black boxes''~\cite{lg-access19,yyx-if22,ya-lim19,carab-ais20}
owing to their opaque internal mechanisms and 
the intricate structure that significantly complicates model interpretation,
making it challenging for humans to understand.
%
With the growing adoption of complex ML models, there is a rising need for
transparency and accountability~\cite{rai-jams20,vw-pt21,lh-ipr20,trai-nature20} 
in their decision-making processes.
%
Users, stakeholders, and regulators require insight into the
rationale behind the decisions or recommendations made by ML/AI systems.
%
Lacking such transparency and accountability can lead to several critical issues
as follows.

\begin{itemize}
	\item Fairness and Bias: 
		%
		Due to the potential for ML/AI systems to generate biased or unfair 
		decisions~\cite{propublica16,hhll-corr22}, explainability is essential for understanding the
		mechanisms behind these biases and mitigating them in ML models. 
		%
		This guarantees that AI-based decisions do not worsen social disparities.
	\item Regulatory Compliance:
		%
		In some sectors, including law and finance, regulations and standards mandate 
		decision-making processes to be transparent and comprehensible.
		%
		The absence of transparency can result in compliance challenges because stakeholders 
		may find it difficult to understand and justify the decisions made by AI 
		systems~\cite{jiv-nmi19,sbbl-aes20,coeck-mitp20}.
	\item Safety and Robustness:
		%
		In safety-critical domains like autonomous vehicles and healthcare, 
		it is crucial to gain insight into how AI systems handle unexpected situations.
		%
		Lacking transparency presents considerable challenges in guaranteeing the safety and
		robustness of AI systems, especially in contexts where the operation of AI models 
		can potentially trigger unexpected consequences~\cite{dear-alr19}.
\end{itemize}

In light of the identified critical issues in ML, there is an imperative demand
to build trust in ML/AI systems.
%
This need has led to rapid development in the research field of \emph{eXplainable AI}~(XAI),
which is aimed at making a bridge between the internal mechanisms of ML/AI systems and human comprehension, 
ultimately targeting establishing \emph{trustworthy AI}~\cite{ms-rw22}.


%In response to the current state of affairs and driven by recent regulations and
%recommendations~\cite{hlegai20,oecd21}, along with existing proposals for 
%ML/AI system regulation~\cite{euaiact21,unesco21,vg-nmi21,radem-rai20},
%there is an urgent need to build trust in the operation of ML/AI systems. This demand has spurred
%rapid growth in the research domain of \emph{eXplainable AI}~(XAI). XAI can be defined as the
%process of bridging the gap between the inner workings of ML/AI systems and human understanding, all
%with the goal of establishing trustworthy AI~\cite{ms-rw22}.

\textbf{The significance of XAI}.
%
The significance of both XAI and trustworthy AI is emphasized by 
recent regulations and guidelines from influential entities~\cite{darpa-xai16,hlegai19,hlegai20,eucoord21,euaiact21,oecd21,unesco21,ms-rw22}.
%
The main reasons of the significance of XAI are identified 
as follows.

\begin{itemize}
	\item Reducing Bias:
		%
		XAI can contribute to identifying and alleviating bias
		in ML models.
		%
		With the assistance of XAI in gaining insight of the features and data points
		affecting decisions, professionals can tackle biases and achieve fairness in AI applications.

	\item Building Trust:
		%
		Trust is a vital factor to consider in the adoption of AI.
		%
		XAI enables stakeholders to obtain a better understanding of the decision-making process, 
		therefore enhancing trust in the technology.
		%
		This is especially vital in situations where AI automates or assists decision-making processes, 
		such as in finance, where investors depend on AI-assisted investment advice, 
		and in healthcare, where doctors require trust in AI-supported diagnoses.
	\item Boosting productivity:
		%
		Methods facilitating explainability can quickly uncover errors and/or areas
		requiring improvement, simplifying tasks for ML operations~(MLOps) teams who are responsible for 
		efficiently monitoring and maintaining AI systems.
		%
		For instance, the insight of particular features that drive the model's output 
		enables technical teams to verify whether the patterns identified by the model 
		have wide applicability and relevance for future predictions,
		or if they merely represent a particular data point.
	\item Enhancing Safety:
		%
		In safety-critical fields, XAI facilitates users
		understanding the rationale behind the decision made by
		AI systems.
		%
		For instance, in the context of autonomous vehicles,
		the understanding of driving system's actions can
		prevent accidents.
	\item Aiding Compliance:
		%
		There is a rising call for transparency in ML algorithms,
		with governments and regulatory organizations increasingly paying 
		attention to AI ethics.
		%
		Aligning with these principles, XAI can assist organizations
		in complying with established regulations and ethical standards.
\end{itemize}

\textbf{XAI Approaches.}
In recent years, various approaches to XAI have emerged, aiming to improve the 
transparency and trustworthiness of ML systems. 
%
In general, XAI techniques can be categorized based on different criteria,
namely intrinsic, post-hoc, model-agnostic, and model-specific methods.

\begin{itemize}
	\item \emph{Intrinsic} approaches concentrate on structuring ML models in a manner that 
		ensures they are inherently interpretable from the beginning.
		%
		In general, models with straightforward architectures are referred to as
		\emph{interpretable} ML models, frequently deemed intrinsically interpretable~\cite{rudin-natmi19}.
		%
		Arguably, the most explainable types of ML models include decision trees,
		decision lists, and decision sets, as they consist of straightforward logical rules.
		%
		Among these, decision sets offer the most straightforward explanations,
		where if a rule in a decision set ``fires'' for a specific data instance, 
		that rule alone serves as the explanation.
		%
		Regarding decision lists, which consist of an ordered sets of rules,
		we are required to additionally consider the order of rules in the model.

	\item \emph{Post-hoc} methods explain trained ML models by applying interpretation
		techniques.
		%
		These techniques are applicable to complex and inherently non-interpretable models, 
		including neural networks and transformers.
		%
		The two primary branches of research in post-hoc explanations include
		\emph{feature selection} techniques, exemplified by Anchors~\cite{guestrin-aaai18}, 
		and \emph{feature attribution} methods, such as LIME~\cite{guestrin-kdd16} and SHAP~\cite{lundberg-nips17}.

	\item \emph{Model-agnostic} techniques~\cite{guestrin-kdd16,lundberg-nips17,guestrin-aaai18} are 
		applicable to any ML model, including complex models.
		%
		These techniques are often implemented after the model has finished its
		training phase, rendering them post hoc in nature.
		%
		Model-agnostic approaches often generate explanations by examining 
		the relationships between inputs and outputs, without depending on access to 
		the internal information of the model, such as weights and structural details.
		%
		Undoubtedly, model-agnostic methods are viewed as the prevailing approach 
		in the domain of XAI.
	
	\item \emph{Model-specific} methods are customized for specific categories of models
		and are not universally applicable.
		%
		In contrast to model-agnostic approaches, model-specific methods can offer more
		detailed insights into particular types of models, although their applicability is 
		restricted to those particular models.
		%
		There is an increasing tendency to apply formal approaches for ML systems
		verification~\cite{sss-cacm22}, with logic-based explainability playing a crucial role in 
		this trend~\cite{ignatiev-ijcai20,msi-aaai22,darwiche-lics23,ms-rw22,msi-fai23}.
\end{itemize}

To gain a deeper insight into these XAI approaches, interested readers can refer to
reference~\cite{asjaa-dmkd21,molnar-bk20,rai-jams20,darwiche-lics23,xudfzz-nlpcc19,hsmbs-xxai22,ignatiev-ijcai20,ms-rw22,msi-fai23}.
\pjs{Why these ones are chosen, are they surveys? If so say so!}
%
The majority of XAI methods explored in this thesis can be classified as intrinsic
and post-hoc approaches.

\section{Motivations of Formal Explainability}

While model-agnostic methods are prevalent in the XAI domain, 
they encounter challenges regarding the quality of explanations. 
%
An emerging alternative is formal explainability, which represents a
rising trend in applying automated reasoning techniques to explain and 
verify ML models.
%
The challenges faced by model-agnostic methods 
and the emergence of formal explainability 
are outlined below.

%\subsection{Limitations of Model-agnostic Explanations}

\textbf{Limitations of Model-agnostic Explanations}.
Numerous XAI techniques introduced in recent years are 
model-agnostic, yet they unfortunately encounter several issues.
%
These issues include out-of-distribution sampling~\cite{lakkaraju-aies20a,lb-aies20,yis-corr23,yfis-corr23}
and unsoundness in generated explanations~\cite{inms-corr19,nsmims-sat19,ignatiev-ijcai20,msi-aaai22}.
%
Due to the limitations presented by these non-formal XAI methods,
a notable challenge regarding the reliability of model-agnostic explanations is identified, 
particularly in high-risk or safety-critical environments~\cite{rudin-nature22,rudin-natmi19,vw-pt21,raai-sp19,dmbt-comp17,hcw-chb21,ms-rw22}.
%
Relying on such unsound explanations generated by model-agnostic methods can trigger severe consequences.
%
In addition to unsoundness, other drawbacks of model-agnostic explanations have been identified~\cite{cgflb-corr19,dbjw-ecai20,kvsf-icml20,jmb-aistats20,hms-corr23,hms-corr23b,hms-corr23c}.

\textbf{Formal Explainability}.
%
Formal XAI~(FXAI) approaches, providing \emph{logic-based explanations} or 
\emph{formal explanations},
stand as an alternative to  model-agnostic methods.
%
These formal XAI methods are characterized by model-specific
and logic-based XAI methods~\cite{ignatiev-ijcai20,msi-aaai22,darwiche-lics23,ms-rw22}.
%
FXAI is aimed at offering rigorous and provable explanations for 
outputs made by ML models.
%
There are two categories of formal explanations, namely,
\emph{abductive explanations}~(AXp’s) and \emph{contrastive explanations}~(CXp’s).
%
AXp's provide answers for \emph{why} a prediction was made, 
while CXp's answer questions of \emph{why not} another prediction was made
or \emph{how} to change the prediction produced by the ML model. 
%
Several studies introduce formal methods for computing explanations~\cite{amgoud-ecsqaru21,ll-clar21,rcblt-ai21,rcbt-kr20,wgh-aies19}, with one prominent
approach in formal explainability building on abductive reasoning~\cite{inms-corr19,inms-aaai19,ignatiev-ijcai20,msgcin-nips20,ims-sat21,msi-aaai22,imsns-ijcai21,hiicams-aaai22,barcelo-nips22,hiims-kr21,hims-aaai23,huang-tacas23,ims-ijcai21,msgcin-icml21,yisnms-aaai23,snimmsv-aaai22,nsmims-sat19}.

%\subsubsection{Compilation-based computation of formal explanations.}
%This approach compiles the decision function of a given ML model into a tractable circuit [320]. The obtained circuit is tractable and replicates the input-output behavior of the ML models. The tractability, i.e. efficient representation and manipulation, of the resulting circuits solely depends on the queries and transformations [99, 341] they can supports. Once the target tractable circuit is obtained, this method can be highly effective. By examining the resulting circuit, formal explanations for any instance become easily accessible [320]. However, in practice, the process of compilation is worst-case exponential in both time and space [72, 178].

\textbf{Abductive Reasoning for Formal Explanations}.
%
In general, abduction-based approaches involves encoding a given
ML model into a logical-based representation.
%
Therefore, the efficiency of generating explanations through
these methods relies on optimizing the encoding from ML models 
to logic-based representations, along with the capabilities of automated 
reasoning tools, e.g. SAT~(Boolean satisfiability), SMT~(satisfiability modulo theories), 
and QBF~(quantified Boolean formula) solvers.
%
However, recent research has presented a technique that uses diverse 
robustness tools to generate formal explanations~\cite{hms-corr23d}, 
which does not explicitly demand the encoding from ML models 
to logic-based representations.

\textbf{Explainability for Interpretable Models}.
%
Intrinsically interpretable ML models~\cite{molnar-bk20,rudin-nmi-19,blrs-ceur22,lzlsr-neurips22,clrsww-dss22,srp-fat22}, 
as their name implies, can offer explanations without requiring extra computation.
%
Decision trees~\cite{rivest-ipl76,breiman-bk84,quinlan-bk93,quinlan-ml86}, 
decision lists~\cite{rudin-mpc18,rivest-ml87}, 
and decision sets~\cite{leskovec-kdd16,michalski-isip69} are widely considered as 
the most interpretable ML models.
%
Recent studies have revealed that formal explanations for decision trees 
can be notably more compact than explanations offered by interpretable ML models~\cite{iims-corr20,iims-jair22},
with decision tree explanations aligning with the paths within the tree.
%
Moreover, the interpretability of decision lists and sets presents challenges~\cite{msi-fai23}. 
%
These findings indicate that even interpretable models still require explanation or 
further interpretability improvement.


\section{Research Questions and Contributions}

\subsection{Research Questions}
Motivated by the rapid development in the XAI domain
and the identified limitations of prevalent model-agnostic methods,
this thesis focuses on XAI problems, in particular formal explainability, 
aiming to develop techniques to improve explainability of ML models 
and address the gap in formal explainabaility.
%
The thesis is categorized into three research questions (RQs):
\begin{itemize}
	\item RQ1: How can we learn ML models that are both accurate and interpretable?
	\item RQ2: How can we compute succinct and accurate explanations?
	\item RQ3: How can we apply formal explanations in real-world scenarios? 
\end{itemize}

\subsection{Contributions}

\subsection{Contributions for RQ1.}

To address RQ1, two research works are introduced:

\begin{itemize}
	\item Jinqiang Yu, Alexey Ignatiev, Peter J. Stuckey, and Pierre Le Bodic. Learning optimal decision
sets and lists with sat. \emph{Journal of Artificial Intelligence Research,} 72, 1251-1279, 2021.

	\item Jinqiang Yu, Alexey Ignatiev, and Peter J. Stuckey. From formal boosted tree explanations to
interpretable rule sets. \emph{In 29th International Conference on Principles and Practice of
	Constraint Programming,} vol. 280, pp.38:1-38:21, 2023.
\end{itemize}

First, we present a method for constructing decision sets and decision lists optimal in
either the number of required literals or the trade-off between model size and accuracy.
%
In this study, we introduce a novel metric for assessing the explainability of 
interpretable ML models, specifically focusing on the number of literals used.
%
Previous research has primarily relied on a metric based on the number of rules, 
which may not accurately capture explainability.
%
For instance, two rules with 70 conditions each are considerably
less interpretable than six rules with only three conditions each.
%
With the new explainability measure, we can generate more interpretable decision sets and lists.

Furthermore, we introduce a novel anytime approach to
generating decision sets that focus on both
accuracy and interpretability.
%
Motivated by the issue in existing approaches to decision sets,
which cannot offer any decision information 
if a dataset is not completely solved,
we propose a method that establishes a connection
between formal post-hoc explainability and interpretable decision set models.
%
This involves distilling a gradient boosted tree model 
into a decision set as needed,
making use of formal explanations in the process.
%
This technique is a knowledge distillation method and in theory, it can be extended
to other ML models.


\subsection{Contributions for RQ2.}

To tackle RQ2, three research works are proposed: 

\begin{itemize}

	\item Jinqiang Yu, Alexey Ignatiev, Peter J. Stuckey, Nina Narodytska, and Joao Marques-Silva.
Eliminating the impossible, whatever remains must be true: On extracting and applying background
knowledge in the context of formal explanations. \emph{In Proceedings of the AAAI Conference on Artificial
Intelligence,} vol. 37, pp. 4123-4131, 2023.

	\item Jinqiang Yu, Alexey Ignatiev, and Peter J. Stuckey. On Formal Feature Attribution and Its
Approximation. \emph{arXiv preprint arXiv:2307.03380,} 2023.

	\item Jinqiang Yu, Graham Farr, Alexey Ignatiev, and Peter J. Stuckey. Anytime Approximate Formal
Feature Attribution. \emph{arXiv preprint arXiv:2312.06973,} 2023.
\end{itemize}


We introduce a method for leveraging background knowledge,
i.e. correlations between features, to generate more concise "why" formal explanations
which are presumably more understandable to humans,
and to provide more precise "why not" explanations.
%
These approaches tackle the problem of lengthy formal explanations, 
which are caused by the general formal methods that consider the entire 
feature space under the assumption of feature independence and
uniform distribution~\cite{kutyniok-jair21}.
%
This issue can make the explanations difficult for users to interpret, 
and also limit the practical applicability of the formal explainability
techniques.

In addition to the drawback of formal explanations being lengthy,
the formal approach also fails to provide feature attribution.
%the formal approach also fails to address the issue of feature attribution.
%
To overcome this limitation, we propose a new formal method for feature attribution.
%
By exhaustively enumerating all formal explanations, we establish a clear definition
of formal feature attribution~(FFA) as the proportion of explanations
where a particular feature appears.
%
However, we argue that formal feature attribution presents challenges for
the second level of the polynomial hierarchy.
%
Therefore, we further extend our work by developing an anytime approach 
that enables the efficient computation of approximate FFA 
and thus extending its practical applicability.

\subsection{Contributions for RQ3.}

To solve RQ3,  one research work is presented:
\begin{itemize}
	\item Jinqiang Yu, Michael Fu, Alexey Ignatiev, Chakkrit Tantithamthavorn, and Peter J. Stuckey. A
Formal Explainer for Just-In-Time Defect Predictions. \emph{ACM Transactions on Software Engineering
	and Methodology,} 2024. (\textbf{Accepted.}) 
\end{itemize}


We propose an method to apply formal explainability in a just-in-time (JIT) defect prediction.
%
JIT defect prediction aims to predict whether a commit will potentially introduce
software defects, helping teams in directing their limited resources towards the
most high-risk commits or pull requests.
%
This proposed approach offers explanations that are not only provably correct,
but also guaranteed to be minimal.
%
This tackles the shortcomings observed in previous research, where model-agnostic techniques are
used to explain JIT model predictions, resulting in explanations
that are not formally sound, robust, and actionable.
%
Our proposed approach is capable of efficiently producing explanations that 
are proven to be correct, robust, and actionable, while existing model-agnostic techniques
lack this capability.


\section{Thesis Organization} \label{sec:structure}
The thesis is organized as follows:
\begin{itemize}
	\item \autoref{chap:bg} reviews the background related to XAI.
	\item \autoref{chap:jair21} introduces the method to generate optimal decision sets and list.
	\item \autoref{chap:cp23} shows the work to compile a gradient boosted tree into a
		decision set.
	\item \autoref{chap:aaai23} presents the technique to apply background to formal
		explainability.
	\item \autoref{chap:ffa}  illustrates the approach to computing formal feature attribution.
	\item \autoref{chap:marco}  shows the anytime method to approximate formal feature
		attribution.
	\item \autoref{chap:jit} introduces work to apply formal explainability in just-in-time
		prediction.
	\item \autoref{chap:conc} presents the conclusions and feature work.
\end{itemize}

%This dissertation is organised as follows:
%• Chapter 2 reviews the literature that related to pathfinding problem.
%• Chapter 3 presents our techniques [45, 46] to efficiently find optimal and suboptimal obstacle-avoiding path in a Euclidean plane.
%• Chapter 4 covers our efficient oracle-based approach [47] for retrieving the shortest path in a static road network.
%• Chapter 5 describes our enhancement techniques [49] on the state-of-the-art algorithms, TCH, for finding the shortest path in a time-dependent road network.
%• Chapter 6 shows our novel cluster reasoning techniques [50] to enhances the performance of the CBS algorithm for solving the classic MAPF problem.
%• Chapter 7 describes our proposed platform [51] for continuously monitoring the progress of MAPF algorithms.
%• Chapter 8 concludes our research and discusses future works.
