

%\begin{frame}{Anytime Approximate Formal Feature Attribution~(RQ2)}
%	\begin{itemize}
%		\item AXp's diverse then we can target AXp's
%
%		\item FFA is proposed to feature attribution based heuristic approaches but hard to
%			compute %, as deciding if a feature has a non-zero attribu- tion is at least as hard as deciding feature relevancy
%
%		\item recalled that FFA can be efficiently computed by making use of the hitting set duality between AXp’s and CXp’s. By trying to enumerate CXp’s, a side effect of the algorithm is to discover AXp’s.
%
%		\item issue: In fact, the algorithm will usually find many AXp’s before finding the first CXp. 
%
%		\item The AXp’s are guaranteed to be diverse, since they need to be broad in scope
%			to ensure that the CXp is large enough to hit all AXp’s that apply to the
%			decision.
%
%		\item observations: Using AXp’s collected as a side effect of CXp enumeration is effective at the
%			start of the enumeration.
%
%		\item But as we find more and more AXp’s as side effects we eventually get to a point where many more CXp’s are generated than AXp’s.
%		
%		\item Experimentation: enumerate all AXp’s then indeed we should not rely on the
%			side effect behavior, but simply enumerate AXp’s directly. This leads to a
%			quandary: to get fast accurate approximations of FFA we wish to enumerate
%			CXp’s and generate AXp’s as a side effect. But to compute the final (exact) correct FFA we wish to compute all AXp’s, and we are better off directly enumerating AXp’s.
%
%		\item our approach: we develop an anytime approach to computing approximate FFA, by
%			starting with CXp enumeration, and then dynamically switching to AXp
%			enumeration when the rate of AXp discovery by CXp enumeration drops. (switch
%			when meeting criteria)
%
%		\item In doing so, we are able to quickly get accurate approxima- tions, but also arrive to the full set of AXp’s quicker than pure CXp enumeration. 
%
%		%As direct CXp enumeration is fea- sible to do without the need to resort to the hitting set dual- ity (Marques-Silva and Ignatiev 2022), one may want to esti- mate FFA by first enumerating CXp’s. The second contribu- tion of this paper is to investigate this alternative approach and to show that even if a(n) (in)complete set of CXp’s is given, determining FFA is computationally expensive being #P-hard even if all CXp’s are of size two.
%	\end{itemize}
%
%\end{frame}

\begin{frame}{Anytime Approximate Formal Feature Attribution~(RQ2)}
	\textbf{Issue}: Exact FFA hard to compute.

	\textbf{Observations}: 
	\begin{itemize}
		\item FFAs are computed by collected AXp's as a side effect of CXp
	enumeration.
		\item Usually find many AXp’s before finding the first CXp. 

		\item AXp’s are diverse. %Need to be broad to ensure that the CXp is large enough to hit all AXp’s.
		
		\item Diverse AXp's $\rightarrow$ good approximation of FFA

		\item AXp enumeration can get\emph{exact} FFA faster% $\emph{exact}$
	\end{itemize}
	
	\textbf{Our approach}: 
		\begin{itemize}
			\item Anytime approach to computing approximate FFA
			\item Start with CXp enumeration
			\item Switch to AXp enumeration at some point.
		\end{itemize}

\end{frame}

\begin{frame}{Algorithm with switching}
	\input{switchalg}
\end{frame}


\begin{frame}{Switching Criteria}
	\small 

	Observe that normally $|\axp| > |\cxp|$
	
	\textbf{Criterion 1}:
	Switch when CXp's on average are \emph{much} smaller than AXp's, i.e. when
	\begin{align}
	    \frac{\sum_{\fml{X}\in\axps^w}{|\fml{X}|}}{\sum_{\fml{Y}\in\cxps^w}{|\fml{Y}|}} \geq \alpha,
	\end{align}
	\textbf{Criterion 2}:
	Switch when the average CXp size ``stabilizes''.
	\begin{align}
	    \left|\left|\fml{Y}_{\text{new}}\right|-\frac{\sum_{\fml{Y}\in\cxps^w}{|\fml{Y}|}}{w}\right| \leq \varepsilon,
	\end{align}
	\textbf{Rational}: 
	\begin{itemize}
		\item CXp extraction: check satisfiable; cheap 
		\item AXp extraction: check unsatisfiable, expensive
		\item Before switching: ensure AXp's diverse
		\item After switching: single call for AXp, multiple calls for CXp extraction
	\end{itemize}

	%Given window size~$w$, switching criteria build on %for both axp's and cxp's, both criterio build on 
	%\begin{itemize}
	%	\item the average \emph{size} of the last $w$ AXp's %and
	%	\item the size of last $w$ CXp's enumerated % in the most recent iterations of the MARCO algorithm.
	%\end{itemize}
\end{frame}

%\begin{frame}{Criterion 1}
%Switch when CXp's on average are \emph{much} smaller than AXp's, i.e. when
%
%\begin{align}\label{eq:cond1}
%    \frac{\sum_{\fml{X}\in\axps^w}{|\fml{X}|}}{\sum_{\fml{Y}\in\cxps^w}{|\fml{Y}|}} \geq \alpha,
%\end{align}
%
%where $\axps^w$/ $\cxps^w$:  last $w$ AXp's/ CXp's
%%\begin{itemize}
%%	\item $\axps^w$/ $\cxps^w$:  last $w$ AXp's/ CXp's
%%	\item $\alpha\in\mbb{R}$: predefined parameter
%%\end{itemize}
%
%\textbf{Rational}: 
%	%Recall that extraction of a subset-minimal dual explanation is done by deciding the
%	%validity of the corresponding predicate, either \eqref{eq:axp} or \eqref{eq:cxp}, while iteratively
%	%removing features from the candidate feature set.
%	%
%	%As such, if the vast majority of CXp's is small, their extraction leads to the lion's share of the decision oracle calls being \emph{satisfiable}.
%	%
%	%On the contrary, extracting large AXp's as dual explanations leads to most of the oracle calls proving \emph{unsatisfiability}.
%	%
%	%Hence, we prefer to deal with cheap satisfiable calls rather than expensive unsatisfiable ones.
%	\begin{itemize}
%		\item CXp extraction: Check satisfiable; cheap 
%		\item AXp extraction: check unsatisfiable, expensive
%		\item Before switching: ensure AXp's diverse
%		\item After switching: single call for AXp, multiple calls for CXp extraction
%	\end{itemize}
%\end{frame}
%
%\begin{frame}{Criterion 2}
%Switch when the average CXp size ``stabilizes''.
%%
%%Here, let us denote a new CXp being just computed as $\fml{Y}_{\text{new}}$.
%%%
%%Then the second criterion can be examined by checking if the following holds:
%
%\begin{align}\label{eq:cond2}
%    \left|\left|\fml{Y}_{\text{new}}\right|-\frac{\sum_{\fml{Y}\in\cxps^w}{|\fml{Y}|}}{w}\right| \leq \varepsilon,
%\end{align}
%
%\begin{itemize}
%	\item $\fml{Y}_{\text{new}}$: a new CXp
%	\item $\cxps^w$:  last $w$ CXp's
%\end{itemize}
%%with $\varepsilon\in\mbb{R}$ being another numeric parameter.
%
%\textbf{Rational}: 
%\begin{itemize}
%	\item Signal: AXp's diverse enough for all the CXp's to be of roughly equal size
%	\item Diverse AXp's $\rightarrow$ good approximate FFA
%\end{itemize}
%
%%Overall, the switching can be performed when either of the two conditions \eqref{eq:cond1}--\eqref{eq:cond2} is satisfied.
%\end{frame}



\begin{frame}{Experimental Setup}
	
	\textbf{Datasets}: 3 Images and 2 text data
	
	\textbf{Metrics}: 
	\begin{itemize}
		\item \textbf{Errors}:  Manhattan distance, i.e. the sum of absolute differences across all features. 
			\item \textbf{Kendall's Tau}: Similarity of two rankings. 
			Ranging $[-\text{1}, \text{1}]$.
			The higher the closer.

		%\item \textbf{Kendall's Tau}/ \textbf{RBO}: Similarity of two rankings. 
		%	Ranging $[-\text{1}, \text{1}]$ and $[\text{0}, \text{1}]$. 
		%The higher the closer.
	\end{itemize}
	
	\textbf{Average Runtime}: 
	\begin{itemize}
		\item MARCO-S (Our approach): 3509.50s (9.26s $-$ 30881.55s)
		\item MARCO-A (AXp enumeration): 3255.30s (2.15s $-$ 29191.42s)
		\item MARCO-C (CXp enumeration): 19311.87s (9.39s $-$ 55951.57s)
	\end{itemize}
\end{frame}

\begin{frame}{Error Results}
	\begin{figure}[!t]
	\centering
	    \includegraphics[width=0.6\textwidth]{plots/switch/switch-slide50-gap2-diff1-normalise2-error}
	    \caption{FFA error over time.} 
	    \label{fig:error}
	\end{figure}
		{\large\adfbullet{9} \color{orange}\normalsize MARCO-S}: Propose approach
		{\large\adfbullet{9} \color{orange}\normalsize MARCO-A}: AXp enumeration  \\   
		{\large\adfbullet{9} \color{orange}\normalsize MARCO-C}: CXp enumeration
\end{frame}

\begin{frame}{Kendall's Tau Results}
	\begin{figure}[!t]
		\centering
	    \includegraphics[width=0.6\textwidth]{plots/switch/switch-slide50-gap2-diff1-coef-kendalltau}
	    \caption{Kendall's Tau over time.}
	    \label{fig:tau}
	     \end{figure}
		{\large\adfbullet{9} \color{orange}\normalsize MARCO-S}: Propose approach
		{\large\adfbullet{9} \color{orange}\normalsize MARCO-A}: AXp enumeration  \\   
		{\large\adfbullet{9} \color{orange}\normalsize MARCO-C}: CXp enumeration
\end{frame}

%\begin{frame}{RBO Results}
%	\begin{figure}[!t]
%		\centering
%	    \includegraphics[width=0.6\textwidth]{plots/switch/switch-slide50-gap2-diff1-coef-rbo}
%	    \caption{RBO over time.}
%	    \label{fig:rbo}
%\end{figure}
%		{\large\adfbullet{9} \color{orange}\normalsize MARCO-S}: Propose approach
%		{\large\adfbullet{9} \color{orange}\normalsize MARCO-A}: AXp enumeration  \\   
%		{\large\adfbullet{9} \color{orange}\normalsize MARCO-C}: CXp enumeration
%\end{frame}

