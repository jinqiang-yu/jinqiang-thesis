\chapter{Conclusions and Future Work}\label{chap:conc}

\section*{Conclusions}

% growing XAI and its application in various domains
% model-agnostic and FXAI
%

While machine learning~(ML) and Artificial Intelligence~(AI) advancements have been 
widely applied across diverse domains like finance, 
the lack of transparency in these systems triggers significant concerns, 
e.g. fairness, bias, safety and robustness.
%
As a result, there is a growing interest of eXplainable AI~(XAI) aimed to establish trust in ML/ AI 
systems, by explaining or illustrating the behavior of ML models in 
human-understandable ways.
%
Among various approaches to XAI, model-agnostic approaches
stand out as the prevailing ones although they encounter challenges regarding 
the quality of generated explanations. 
%
As an alternative, formal XAI~(FXAI) methods offer logic-based or formal explanations, 
aimed to provide provable and robust explanations for ML predictions.
%
This thesis concentrates on XAI challenges, especially formal explainability,
targeting enhancing the interpretability of ML models
and addressing gaps in formal explainability.
%
The primary contributions of this thesis are outlined as follows:

% contributions again
\begin{itemize}
	\item \autoref{chap:jair21} introduces methods to generate decision sets and decision lists 
		that are optimal either in terms of the number of required literals or 
		the trade-off between model size and accuracy.
		%
		In this study, we define size as the total number literals used in these rule-based ML models,
		in contrast with previous research that primarily focuses on the number of rules in
		the model, which cannot adequately capture the explainability of such models.
		%
		The empirical results illustrate that the decision sets and lists computed by the
		proposed approaches are able to achieve decent trade-off between interpretability (represented by model size) and accuracy.

	\item In \autoref{chap:cp23}, we present an innovative anytime approach to producing decision sets
		through the on-demand extraction of generalized abductive explanations for boosted
		trees.
		%
		The proposed method can be used to compile a gradient boosted tree
		with respect to either the complete feature space, or a set of
		target training instances.
		%
		Augmented by post-hoc model size reduction approaches, 
		this method is demonstrated to generate decision sets that outperform those
		computed by state-of-the-art algorithms in terms of accuracy accuracy and remain 
		comparable with them regarding explanation size.
		%
		Note that the presented method can be categorized as a knowledge distillation technique and theoretically, 
		it can extend to any other ML models.

	\item In \autoref{chap:aaai23}, we propose a technique to apply background knowledge to enhance 
		the quality of explanations. 
		%
		There are substantial advantages of incorporating background knowledge in generating
		formal explanations for ML models.
		%
		In the context of abductive explanations~(AXp's), 
		integrating background knowledge significantly reduces the length of explanations, 
		making them more explainable, and improve the efficiency of explanation generation.
		%
		In the case of contrastive explanations (CXps), although applying background knowledge may 
		increase the explanation size and potentially require more time for explanation generation, 
		the resulting explanations are significantly more precise because they do not 
		depend on the (often unsupportable) assumption that all tuples in the feature space 
		are feasible.
		%
		Moreover, as demonstrated in \autoref{chap:aaai23}, background knowledge can be 
		integrated into the context of heuristic explanations, particularly when an accurate
		analysis of is need.

	\item \autoref{chap:ffa}  introduces the first method for formal feature attribution~(FFA),
		using the proportion of abductive explanations where a feature appears to
		indicate its significance.
		%
		Experimental results demonstrate that we can compute exact FFA for numerous classification tasks, 
		and in cases where exact computation is infeasible, we can compute effective approximations.
		%
		Additionally, we found existing model-agnostic method to generate to feature attribution 
		disagree with FFA.
		%
		In some cases, they are considerably different from FFA, such as assigning no weight to 
		a feature that is present in (a significant number of) explanations,
		or assigning a (large) non-zero weight to a feature that holds no relevance for the prediction.
		%
		Overall, \autoref{chap:ffa} argues that if we acknowledge FFA as a valid measure 
		of feature attribution, it is important to explore approaches that generate good 
		approximate FFA more efficiently.

	\item In \autoref{chap:marco}, motivated by the difficulty of exact computation for 
		many classifiers and datasets, we introduce an anytime method for FFA
		approximation.
		%
		As illustrated in this chapter, computing FFA remains challenging even 
		when the set of CXp's is available. 
		%
		Therefore, there is a demand for anytime method to generate FFA.
		%
		Surprisingly, starting with CXp enumeration to produce AXps results in 
		rapidly achieving good approximations of FFA.
		%
		However, in the longer turn this method proves to be less effective 
		compared to simply enumerating AXps.
		%
		This work demonstrates the integration of these approaches by strategically 
		switching the enumeration phase, ensuring that information computed in the 
		underlying MARCO enumeration algorithm is preserved.
		%
		This presents a highly practical method to generate FFA.

	\item \autoref{chap:jit} introduces the first formal explainer for just-in-time~(JIT) 
		defect prediction. 
		%
		This novel explainer leverages formal reasoning by exploiting propositional logic
		to efficiently produce explanations that are provably correct, robust, 
		and subset-minimal, addressing ``why'' questions, such as why a commit is 
		predicted as defective.
		%
		The proposed formal explainer is also capable of producing contrastive ``how'' 
		explanations that are provably correct, robust, and subset-minimal, 
		offering more actionable insights compared to abductive explanations.
		%
		This marks a significant step towards actionable software analytics.
		%
		Therefore, these generated explanations are able to assist researchers 
		in designing actionable defect prediction methods and help practitioner direct their
		attention towards the most crucial aspects related to software defects. 
		%
		This, in turn, enhances the operational decision-making of software quality assurance~(SQA) teams.
\end{itemize}

\section*{Feature work}
While this thesis presents numerous advancements in the field of XAI, 
there are still several directions that need to be addressed in future research.
%
Some future directions are outlined as follows.

\textbf{Applying Formal Explainability}.
%
Although ML and AI are widely used in diverse fields, such as healthcare, law, finance, and transportation, 
the opacity of ML/ AI systems can hinder users from gaining clear insights into 
the outputs produced by these systems.
%
Consequently, these systems fail to establish trust between humans and the
predictions generated by ML models.
%
Inspired by the limitation, in \autoref{chap:jit} we \newm{apply} formal explainability for
just-in-time~(JIT) defect prediction, aiming to establish trust for the predictions.
%
To broaden the applicability of ML and AI applications, one future research direction 
will focus on applying formal explainability across various fields.
%
This will help users trust the predictions generated by ML models, facilitating the wider 
adoption of ML and AI in diverse domains.

\textbf{Improving Scalability of Formal Explainability}.
%
With the increasing need to deploy large and complex ML/ AI systems across diverse domains, 
the scalability of XAI approaches has emerged as a notable concern.
%
Unfortunately, formal explainability encounters scalability challenges, 
particularly with certain complex classifier families, such as neural networks.
%
These models are computationally expensive or hard to reason formally
to produce formal explanations due to their complex structure.
%
To address this scalability limitation, recent research~\cite{bk-tacas23} 
suggests a novel method to approximate formal explanations, with the use of 
technologies aimed at evaluating the robustness of neural networks.
%
Additionally, recent research~\cite{bk-tacas23} introduces innovative algorithms for computing
formal explanations, achieved by finding a direct relationship between 
the practical complexity of robustness and formal explainability.
%
Inspired by the scalability issue and these studies, one feature research direction
will focus on developing methods to efficiently
generate formal explanations for complex ML models.
