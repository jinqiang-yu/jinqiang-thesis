\section{Classification and Machine Learning Models} \label{sec:ml}

\subsection{Classification Problems} \label{sec:cls}
A classification problem considers a set of features $\feats=\{1, \ldots, m\}$
and a set of classes~\footnote{\newm{Non-integer classes can be mapped in to a set of $k$ numeric
	classes.}} $\clss = \{1, 2, \ldots, k\}$.
%$\footnote{Any set of classes $\{c_1,\ldots,c_k\}$ can
%always be mapped into the set of the corresponding indices
%$\{1,\ldots,k\}$.}
%%
Each domain $\domain_i$ for feature $i \in \feats$ can be
categorical or ordinal, i.e.\ integer, and the value of each
feature $i$ is taken from its corresponding domain~$\domain_i$.
%
Therefore, the entire feature space is defined as
$\cfeats\triangleq\prod_{i=1}^{m}\domain_i$.
%
For boolean domains, $\domain_i = \{0, 1\}$,
$i \in \feats$, and $\feats = \{0, 1\}^m$.
%
The notation $\mbf{v}=(v_1,\ldots,v_m)\in\cfeats$ represents
a specific point in feature space, where each $v_i \in
\domain_i$ is a constant taken for feature $i\in\feats$.
%
An \emph{example} or \emph{instance} is denoted by a concrete point
$\mbf{v} \in\cfeats$ in feature space and its corresponding class $c
\in \clss$, i.e. an instance is represented by a pair $(\mbf{v}, c)$.
%
Given a subset~$\fml{S} \in \feats$, $\mbf{v}_S$ is denoted as the
partial point of $\mbf{v}$, indicating the restriction of the entire
point $\mbf{v}$ to those features in $\fml{S}$.
%
An arbitrary point in feature space is denoted by
the notation $\mbf{x} = (x_1,\ldots,x_m)$, where each $x_i \in \mbf{x}$
is a variable which takes values from its corresponding domain~$\domain_i$
and represents feature $i\in\feats$.
%
Moreover, the set of classes~$\clss$ is assumed to contain
finite classes; no extra restrictions are placed upon $\clss$.
%
Finally, a non-constant classification function $\cfunc:
\cfeats \limply \clss$ is defined by a machine learning classifier~$\model$,
which is represented as a tuple ($\cfeats$, $\feats$, $\clss$, $\cfunc$).

\subsection{Decision Sets \& Lists}
Decision sets~\cite{leskovec-kdd16,michalski-isip69,ipnms-ijcar18,yilbs-cp20}
and decision lists~\cite{rudin-mpc18,rivest-ml87} are considered interpretable
machine learning models, representing families of rule-based classifiers.
%
These models offer users concise explanations directly from the models themselves.

A \emph{decision rule} is structured as ``IF antecedent THEN
prediction'', with the antecedent being a set of feature \newm{conditions}.
%
A rule is considered to assign an instance $\mbf{v}\in\feats$
to class $c\in\clss$ if its antecedent is \emph{compatible} with
$\mbf{v}$ or \emph{matches} $\mbf{v}$ and its prediction is $c$.

A \emph{decision set}~(DS) consists of an unordered set
of decision rules~$\fml{R}$.
%
An instance $(\mbf{v},c)\in\fml{E}$ is misclassified by a DS if either
there are no rules in $\fml{R}$ compatible with $\mbf{v}$, or if there
is at least one rule that classifies $\mbf{v}$ as a class $c'\in\feats$
where $c'\neq c$.

A \emph{decision list}~(DL) is an ordered set of rules.
%
The first rule of a decision list matches an instance is the one classifying the instance.
%
Decision lists are typically represented as as a single cascade of
IF-THEN-ELSEs, with the final rule serving as a default rule,
which assigns all remaining instances to a certain class.

%Compared with DLs, there are a number of issues in DSs
%complicating their analysis.
%
Overlapping is one of the issues \newm{in DSes}, where two or more rules predicting
different classes fire on the same inputs~\cite{ms-rw22,msi-fai23}.
%
Coverage of feature space is another issue, where instances are not
fired by any rule in DSs.
%
To address this coverage problem, a default rule can be added, which
assigns a certain class to instances fired by none of the other rules.
%
However, this condition further complicates the
reasoning process for DSs.

\subsection{Decision Trees and Gradient Boosted Trees}
Decision trees~\cite{rivest-ipl76,breiman-bk84,quinlan-bk93,quinlan-ml86}
are another interpretable machine learning model, often considered to be
more advantageous than most ML models as their decisions are generally
straightforward to understand by humans.

A \emph{decision tree}~(DT) $\tree = (\tnodes, \edges)$ consists of a set of
nodes~$\tnodes$ and a set of edges~$\edges$,
forming a directed acyclic graph where there is at most one
edge between any pair of nodes.
%
The root node of a decision tree~$\tree$ has no incoming edges,
while all other nodes have a single incoming edge.
%
In this thesis, univariate decision trees are considered,
where every non-terminal node is linked to exactly
one feature $i \in \feats$, and each terminal node
corresponds to a value from a set of classes~$\clss$.
%
Furthermore, each non-terminal node signifies a feature
condition in the form of $x_i < d$, where feature $i \in \fml{F}$
and \emph{splitting threshold}~$d \in \domain_i$.
%
Each path in the DT~$\tree$ is represented as a sequence of nodes,
e.g. $\tpath = \langle \tnode_1, \dots, \tnode_n \rangle$,
where $\tnode_1$ is the root node and $\tnode_n$ is the terminal node.
%
Every pair $(\tnode_j, \tnode_{j+1})$ signifies an edge in the tree~$\tree$.
%
In each path~$\tpath \in \tree$ from the root node to a terminal node,
a feature may be tested multiple times, i.e., it is not read-once.
%
An instance is assigned the class linked to the
terminal node in the path that matches this instance.


While decision trees are easily understood by humans,
they are prone to bias when they are small and tend to overfit
when large.
%
To address these shortcomings, tree ensembles were introduced,
which involve generating multiple decision trees and aggregating
their decisions to reach a final decision.
%
Gradient boosted trees is one of the popular ensemble methods used
to overcome the decision tree's limitations.

A \emph{gradient boosted tree}~(BT) is a tree ensemble~$\trees$
constructed by iteratively building a sequence of small
decision trees.
%
At each stage, new trees are added to the ensemble to address
the inaccuracies present in the existing tree ensemble.
%
Concretely, a BT~$\trees$ defines sets of decision trees~$T_{c} \in \trees$
for each class $c \in [|\clss|]$, where $T_{c}$ consists of
$N \in \mathbb{N}_{>0}$ trees~$\tree_{kz+c}$,
$z \in \{0, \dots, N-1\}$, $k = |\fml{K}|$.
%
Thus, each decision tree is associated with a certain class~$c$
and contributes to the weight class~$c$.
%
Given an instance $\mbf{v}\in\cfeats$, its
class is determined by calculating the sum of scores assigned
by trees for each class $w(\mbf{v},c) = \sum_{t \in T_{c}} t(\mbf{v})$ and
assigning the class with the highest score, i.e. $\textrm{argmax}_{c \in
[|\fml{K}]|} w(\mbf{v},c)$.


\subsection{Neural Networks} \label{sec:nn}
While these interpretable machine learning models including decision sets,
decision lists, and decision trees are straightforward to explain their predictions,
they often lack the accuracy achieved by other state-of-the-art models.
%
Neural networks~\cite{schm-nn15}, which are the fundamental structures of
advanced machine learning models, such as convolutional neural
networks~(CNN)~\cite{kun-bc80,lbbh-ieee98}, recurrent neural
networks~(RNN)~\cite{elman-cs90} and transformers~\cite{transformer-17},
can be adopted to address the accuracy performance issues
associated with these interpretable models

A \emph{neural network}~(NN) is a network~$N$ composed of
artificial neurons where an instance $\mbf{v}\in\cfeats$ is
passed through a sequence of layers~$\fml{L} = \{l_1, l_2, \dots, l_n\}$
and each layer~$l\in\fml{L}$ consists of a set of neurons.
%
Here, the first layer $l_1 \in \fml{L}$ is the input layer,
layers~$\{l_2, \dots, l_{n-1}\}$ are the hidden layers, and the final
layer~$l_n$ acts as the output layer.
%
Functionally, a neural network defines a mapping~$N: \cfeats \limply \clss$.
%
In classification tasks, the output yields scores (or probabilities)
across $|\clss|$ classes, with the class~\newm{$c \in
[|\fml{K}]|$} receiving the highest
score being the prediction of the input instance~$\mbf{v}$.
%
Each neuron~$o$ in a neural network~$N$ forms interconnected
nodes arranged in layers.
%
These neurons collaborate to process and convey information.
%
Each neuron~$o$ accepts inputs, processes them, and generates an output,
which is subsequently conveyed to other neurons in the following layer of
the network.
%
In each layer~$l_i \in \fml{L}$, multiple inputs are received from either the
instance~$\mbf{v}$ (for the input layer~$l_1$, i.e. $i=1$) or
the outputs of the neurons in the previous layer~$l_{i-1}$,
$i \in \{2, \dots. n\}$.
%
Each neuron~$o$ in layer~$l_i$ applies weights to them, aggregates them
with a bias term,
and afterwards passes the result through an activation function
to generate an output~\newm{$a_o^{(l_i)}$, i.e.
\begin{equation}
	a_o^{(l_i)} = \sigma(\sum_{j=1}^{n_{i-1}} w_{oj}^{(l_i)} a_j^{(l_{i-1})} + b_o^{(l_i)})
\end{equation}
where
\begin{itemize}
	\item $\sigma$ is the activation function of neuron~$o$ in layer~$o_i$,
		e.g. \newmm{Sigmoid}, ReLU, and tanh.
	\item $n_{i-1}$ is the number of neurons in layer~$l_{i-1}$.
	\item $a_j^{(l_{i-1})}$ is the output of the neuron~$j$ in layer~$l_{i-1}$
	\item $w_{oj}^{(l_i)}$ is the weight of neuron~$o$ in layer~$l_i$ for $a_j^{(l_{i-1})}$.
	\item $b_o^{l_i}$ is the bias term for neuron~$o$ in layer~$l_i$.
\end{itemize}}

%\pjs{Perhaps briefly secribe different activation functions!}
%
Finally, the output layer~$l_n$ in a neural network~$N$ provides scores across
$|\clss|$ classes for an instance~$\mbf{v}$ and assign the class~$c \in |\clss|$
with the highest score.

However, although neural networks can deliver exceptional performance, they often lack
explainability, leading to challenges in understanding their predictions.
%
This highlights the necessity for explainable AI.
