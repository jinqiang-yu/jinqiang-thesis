\section{SAT \& MaxSAT} \label{sec:sat}
Numerous recent studies focusing on computing interpretable ML models~\cite{aaai2021e,ipnms-ijcar18} 
and producing formal explanations~\cite{iisms-aaai22,ims-sat21} have leveraged 
Boolean satisfiability~(SAT) and maximum satisfiability~(MaxSAT) technology. 
%
This section introduces the necessary concepts related to SAT and MaxSAT.
%
Here, we adopt standard definitions for SAT and MaxSAT solving~\cite{sat-handbook}.

\subsection{Boolean Satisfiability~(SAT)}

In a Boolean satisfiability~(SAT) problem~\cite{sat-handbook}, the input 
comprises a propositional formula over a set of propositional variables
with the use of different logical operators on these variables.
%
A propositional formula is considered to be in \emph{conjunctive normal form}~(CNF) 
if it consists of a conjunction~(logical ``and'') of clauses,
where each \emph{clause} is a disjunction~(logical ``or'') of literals. 
%
A \emph{literal} is either a propositional variable~$b$ or its negation~$\neg b$.
%
Whenever convenient, a clause is regarded as a \emph{set of literals}, 
and a CNF formula are considered as a \emph{set of clauses}.
%
A \emph{truth assignment} assigns a value from $\{0,1\}$ to each variable 
in a CNF formula.
%
For a truth assignment, a clause is deemed satisfied if there exists at least 
one of its literals assigned value 1; otherwise, it is falsified by 
the assignment.
%
A formula is considered satisfied if all of its clauses 
are satisfied; conversely, it is falsified if any of its 
clauses are falsified.
%
If there is no assignment satisfying a CNF formula, 
the formula is deemed unsatisfiable.
%
Solving a SAT problem is to determine whether there exists a truth assignment 
satisfying the entire formula, i.e. all clauses in the formula are satisfied.

\subsection{Maximum Satisfiability~(MaxSAT)}

In the case of unsatisfiable formulas,  the maximum satisfiability (MaxSAT) problem is introduced,
which aims to identify a truth assignment that maximizes the number of satisfied clauses.
%
Although numerous variants of MaxSAT are available~\cite[Chapters~23~and~24]{sat-handbook}, 
the thesis is focused on Partial Unweighted MaxSAT, which can be formulated as follows.
%
A formula $\cnf$ is composed of a conjunction of \emph{hard} clauses $\hardcl$ and 
\emph{soft} clauses $\softcl$, i.e. $\cnf=\hardcl\land\softcl$ (or $\cnf=\hardcl\cup\softcl$ in the set
theory notation), where hard clauses $\hardcl$ must be satisfied 
while soft clauses  $\softcl$ indicate a preference for satisfying these clauses. 
%
The aim of the Partial Unweighted MaxSAT problem is to identify a 
truth assignment satisfying all the hard clauses and maximizing the total number of 
satisfied soft clauses.
%
In the examination of an unsatisfiable formula $\cnf$, there is often an interest in 
finding \emph{minimal unsatisfiable subsets} (MUSes) and \emph{minimal correction subsets} (MCSes) of $\cnf$. 
%
These subsets can be defined as follows.

In the analysis of an unsatisfiable formula $\cnf$, one is also often
interested in identifying \emph{minimal unsatisfiable subsets} (MUSes) and
\emph{minimal correction subsets} (MCSes) of $\cnf$, which can be defined as
follows.
%\footnote{The problems we are tackling with these formalisms in
%this paper belong to decidable fragments of first-order logic. While
%the definitions provided here are given for the propositional case,
%their extension to the first-order case is straightforward.}.

\begin{definition}[Minimal Unsatisfiable Subset~(MUS)] \label{def:mus}
  Consider a formula consisting of an unsatisfiable set of clauses~$\cnf=\hardcl\cup\softcl$,
  where $\cnf\entails\bot$.
  %
  A subset of clauses $\mu\subseteq\softcl$ is seen as a \emph{Minimal Unsatisfiable Subset}~(MUS) 
  iff $\hardcl\cup\mu\entails\bot$ and $\hardcl\cup\mu'\nentails\bot$ holds for $\forall{\mu'\subsetneq\mu}$.
\end{definition}
%
Informally, an MUS can be considered as a minimal explanation of the unsatisfiability of a formula~$\cnf$
since it presents the minimal~(irreducible) information required to be augmented to the 
hard clauses~$\hardcl$ in order to achieve unsatisfiability.
%
Alternatively, there may be interest in \emph{correcting} the formula by 
removing some clauses from $\softcl$ to obtain satisfiability.
%
\begin{definition}[Minimal Correction Subset (MCS)] \label{def:mcs}
  Consider a formula comprising an unsatisfiable set of clauses~$\cnf=\hardcl\cup\softcl$, 
  i.e. $\cnf\entails\bot$.
  %
  A subset of clauses $\sigma\subseteq\softcl$ is considered as a 
  \emph{Minimal Correction Subset}~(MCS) iff
  $\hardcl\cup\softcl\setminus\sigma\nentails\bot$ and 
  $\hardcl\cup\softcl\setminus\sigma'\entails\bot$ holds for
  $\forall{\sigma'\subsetneq\sigma}$.
\end{definition}
%
Informally, an MCS can be viewed as the minimal approach to ``correcting'' 
the unsatisfiability of an unsatisfiable formula $\cnf$.
%
One of the key findings in analyzing unsatisfiable CNF formulas is
the MHS
%minimal hitting set (MHS) 
duality relationship between MUSes 
and MCSes~\cite{reiter-aij87,lozinskii-jetai03}.
%
Concretely, given the sets of all MUSes and MCSes of formula~$\cnf$
denoted as $\mbb{U}_\cnf$ and $\mbb{C}_\cnf$ respectively,  
we have $\mbb{U}_\cnf = \mhs(\mbb{C}_\cnf)$ and $\mbb{C}_\cnf = \mhs(\mbb{U}_\cnf)$,
where $\mhs(S)$ represents the minimal hitting sets of $S$, i.e.
$\mhs(S)$ is the minimal sets sharing one element with each subset in $S$.
%
%Formally, $\mins(S) = \{ s \in S ~|~ \forall t
%\subsetneq s,~t \not\in S\}$ identifies the subset-minimal elements of a
%set of sets, $\hs(S) = \{ t \subseteq (\cup S) ~|~ \forall s \in S,~
%t \cap s \neq \emptyset\}$, and $\mhs(S) = \mins(\hs(S))$.
%
This finding of the MHS duality relationship has been widely used 
in developing algorithms for MUSes and MCSes~\cite{stuckey-padl05,liffiton-jar08,lpmms-cj16}, 
and has also applied in various other contexts.
%
In recent years, there has been a surge in the development of novel algorithms 
for extracting and enumerating MUSes and
MCSe~\cite{mpms-ijcai15,bacchus-cav15,lpmms-cj16,mipms-sat16,pmjms-aaai18,lagniez-ijcai18,narodytska-ijcai18,bendik-atva18}.
